\documentclass[11pt,letterpaper]{article}
\usepackage{tikz}
\usepackage[utf8]{inputenc}
\usepackage[LGR, T1]{fontenc}
\usepackage[greek,english]{babel}
\usepackage{alphabeta}
\usepackage{textgreek}
\usepackage{amssymb}
\usepackage{amsmath}
\textwidth 6.5in
\textheight 9.in
\oddsidemargin 0in
\headheight 0in
\usepackage{graphicx}
\usepackage{fancybox}
\usepackage[utf8]{inputenc}
\usepackage{epsfig,graphicx}
\usepackage{multicol,pst-plot}
\usepackage{pstricks}
\usepackage{amsmath}
\usepackage{amsfonts}
\usepackage{amssymb}
\usepackage{eucal}
\usepackage{listings}
\usepackage{xcolor}
\usepackage[left=2cm,right=2cm,top=2cm,bottom=2cm]{geometry}
\pagestyle{empty}
\DeclareMathOperator{\tr}{Tr}
\newcommand*{\op}[1]{\check{\mathbf#1}}
\newcommand{\bra}[1]{\langle #1 |}
\newcommand{\ket}[1]{| #1 \rangle}
\newcommand{\braket}[2]{\langle #1 | #2 \rangle}
\newcommand{\mean}[1]{\langle #1 \rangle}
\newcommand{\opvec}[1]{\check{\vec #1}}
\renewcommand{\sp}[1]{$${\begin{split}#1\end{split}}$$}

\usepackage{lipsum}

\usepackage{listings}
\usepackage{color}

\definecolor{codegreen}{rgb}{0,0.6,0}
\definecolor{codegray}{rgb}{0.5,0.5,0.5}
\definecolor{codepurple}{rgb}{0.58,0,0.82}
\definecolor{backcolour}{rgb}{0.95,0.95,0.92}

\lstdefinestyle{mystyle}{
	backgroundcolor=\color{backcolour},   
	commentstyle=\color{codegreen},
	keywordstyle=\color{magenta},
	numberstyle=\tiny\color{codegray},
	stringstyle=\color{codepurple},
	basicstyle=\footnotesize,
	breakatwhitespace=false,         
	breaklines=true,                 
	captionpos=b,                    
	keepspaces=true,                 
	numbers=left,                    
	numbersep=5pt,                  
	showspaces=false,                
	showstringspaces=false,
	showtabs=false,                  
	tabsize=2
}

\lstset{style=mystyle}

\begin{document}
\pagestyle{plain}

\begin{flushleft}
Σωκράτης Παναγιώτης Βασιλείου\\
ΑΜ: 03121403
\end{flushleft}

\begin{flushright}\vspace{-15mm}
\includegraphics[height=2cm]{ntua_logo.png}
\end{flushright}
 
\begin{center}\vspace{-1cm}
\textbf{\large Θεμελιώδη Θέματα Επιστήμης Υπολογιστών}\\
1η Σειρά Ασκήσεων
\end{center}

 
\rule{\linewidth}{0.1mm}
%%%%%%%%%%%%%%%%%%%%%%%%%%%%%%%%%%%%%%%%%%%%%%%%%%%%%%%%%%%%%%%%%%%%%%%%

\bigskip

\section*{Άσκηση 8}
\definecolor{codegreen}{rgb}{0,0.6,0}
\definecolor{codegray}{rgb}{0.5,0.5,0.5}
\definecolor{codepurple}{rgb}{0.58,0,0.82}
\definecolor{backcolour}{rgb}{0.95,0.95,0.92}

\lstdefinestyle{mystyle}{
    backgroundcolor=\color{backcolour},   
    commentstyle=\color{codegreen},
    keywordstyle=\color{magenta},
    numberstyle=\tiny\color{codegray},
    stringstyle=\color{codepurple},
    basicstyle=\ttfamily\footnotesize,
    breakatwhitespace=false,         
    breaklines=true,                 
    captionpos=b,                    
    keepspaces=true,                 
    numbers=left,                    
    numbersep=5pt,                  
    showspaces=false,                
    showstringspaces=false,
    showtabs=false,                  
    tabsize=2
}

\lstset{style=mystyle}
\begin{lstlisting}[language=Python]
from __future__ import annotations

def horn_sat_solver(horn_formula: list[list[int]], total_literals: int) -> None:
    """
    horn_sat_solver: Given a Horn Formula determines if it's satisfiable

    :param horn_formula: List of Clauses. A Clause is ενψοδεδ as a list of ints.
    
    Examples:
    

    (not(x1) or not(x2)) and (x1 or not(x3)) and (not(x2) or x3) and x2
    horn_formula: [[-1, -2], [1, -3], [-2, 3], [2]]

    :param total_literals: Total number of literals
    """

    score: dict[tuple[int], int] = {}
    # score assigned to each Clause
    for clause in horn_formula:
        score[tuple(clause)] = 0
        for literal in clause:
            if literal < 0:
                score[tuple(clause)] += 1

    reverse_score: dict[int, set(tuple[int])] = {}
    # Clauses assigned to each score
    for clause in score:
        reverse_score[score[clause]] = set()
    for clause in score:
        reverse_score[score[clause]].add(clause)

    appears_as_neg: dict[int, set[tuple[int]]] = {lit: set() for lit in range(1, total_literals + 1)}
    # Clauses that contain a literal's negation
    for clause in horn_formula:
        for literal in clause:
            if literal < 0:
                appears_as_neg[abs(literal)].add(tuple(clause))

    positive_literal: dict[tuple[int], int] = {}
    # Positive literal of each Clause. 
    for clause in horn_formula:
        for literal in clause:
            if literal > 0:
                positive_literal[tuple(clause)] = literal
            else:
                positive_literal[tuple(clause)] = None

    res: set[int] = set()
    while len(reverse_score[0]) != 0:
        current_clause = reverse_score[0].pop()
        literal = positive_literal[current_clause]
        if literal == None:
            print("Unsatisfiable")
            return
        if literal not in res and current_clause not in appears_as_neg[literal]:
            res.add(literal)
            for clause in appears_as_neg[literal]:
                reverse_score[score[clause]].remove(clause)
                reverse_score[score[clause] - 1].add(clause)

    print("Satisfiable")
    sol: dict[int, bool] = {}
    for i in range(1, total_literals + 1):
        sol[i] = True if i in res else False
    print(sol)
\end{lstlisting}
\end{document}
